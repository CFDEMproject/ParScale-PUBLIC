 %\documentclass[doktyp=marbeit,nummern=abschnitt, parskip=half]
\documentclass{article}
\usepackage{geometry}
\geometry{a4paper,left=35mm,right=35mm, top=2.5cm, bottom=2.5cm} 
%\usepackage[utf8]{inputenc}
%\usepackage[T1]{fontenc}

\usepackage{float}
\usepackage{booktabs} %horizontale Linien in Tabellen 
\usepackage{array} %notwendig für neue column-Stile
\usepackage{caption}  
\usepackage{footnote} % footnote cross reference
\usepackage{psfrag}
\usepackage{color}
\usepackage{tikz}
\usetikzlibrary{shapes,arrows}
\usepackage{amssymb}
\usepackage{setspace}
\usepackage{amsmath}
%\usepackage{mathtools}
%\usepackage{breqn}
\usepackage{amsfonts}
\usepackage{amssymb}
\usepackage{graphicx}

\usepackage{subfig}

\newcommand{\RM}[1]{\MakeUppercase{\romannumeral #1}} 

\usepackage{tabularx}
\newcolumntype{L}[1]{>{\raggedright\arraybackslash}p{#1}} 
\newcolumntype{C}[1]{>{\centering\arraybackslash}p{#1}} 
\newcolumntype{R}[1]{>{\raggedleft\arraybackslash}p{#1}}

\usepackage[inactive]{pst-pdf}
\usepackage{pstricks,pst-node,pst-blur}
\newcounter{leaves}
\newcounter{directories}


\newenvironment{directory}[2][0.4\linewidth]%
% Startet neues Verzeichnis 
% und produziert eine Minipage der angeg. Breite.
% Syntax: \begin{directory}[width]{text}
% text muss in eine \parbox der angegebenen Breite passen;
% wenn keine Breite angegeben ist, wird \linewidth angenommen.
{%
\setcounter{leaves}{0}%
\addtocounter{directories}{1}
\edef\directoryname{D\thedirectories}
\begin{minipage}[t]{#1}% <-------- !!!
  \setlength{\parindent}{3\linewidth}
  \addtolength{\parindent}{-\dirshrink\parindent}
  \parskip0pt%
  \noindent
  \Rnode[href=-0.6\dirshrink]{\directoryname}{\parbox[t]{#1}{#2}}%
  \par
}  
{\end{minipage}}
\newcommand{\file}[2][]{%
% Fuer einen einzelnen Eintrag innerhalb der directory-Umgebung.
% Das Argument darf seinerseits eine directory-Umgebung sein.
  \addtocounter{leaves}{1}%
  \edef\leaflabel{L\theleaves\directoryname}%
  \par
  \Rnode{\leaflabel}{\parbox[t]{1\dirshrink\linewidth}{#2\hfill#1}}%
  \ncangle[angleA=270,angleB=180,armB=0,nodesep=1pt]
    {\directoryname}{\leaflabel}%
  % \typeout{\directoryname,\leaflabel}% Debugging
\par}

\newcommand{\dirshrink}{.95}%{.95} 

\usepackage[final]{pdfpages}
\usepackage{listings}
\lstset{showspaces=false,breaklines=true, showtabs=false, numbers=left, numberstyle=\scriptsize, numbersep=5pt,language=[Visual]C++}
\renewcommand{\lstlistingname}{File section}
\usepackage{color}
\usepackage{tikz}
\usetikzlibrary{shapes,arrows}
\bibliographystyle{unsrt}
%#############################################################################################################################

\usepackage[breaklinks, colorlinks=false, pdfborder={0 0 0}]{hyperref}

\clubpenalties 3 10000 10000 100 
\widowpenalty = 10000 
\displaywidowpenalty = 10000 
\usepackage{hyperref}

\begin{document}

\pagenumbering{roman} 

\begin{center}
\LARGE
Analytical Solution for Stefan Diffusion of a wet core
\end{center} 

\section{Mole Fraction Profile}
The mole fraction concentration distribution in the porous spherical shell is:
\begin{align}
y_1(r) = 1 - (1 - y_{1,0}) \left[\dfrac{1 - y_{1,int}}{1 - y_{1,0}} \right]^{\dfrac{r_{core}^{-1} - r^{-1}}{r_{core}^{-1} - R_{part}^{-1}}}
\end{align}

\section{Species Flux}
\begin{align}
\dot{n} = \dfrac{C_{tot} \, D_{eff}}{r^2 \left( \dfrac{1}{r_{core}} - \dfrac{1}{R_{part}}\right)} \, \text{ln} \left( \dfrac{1-y_{1,int}}{1-y_{1,0}} \right)
\end{align}

\section{Drying Time}
The time needed for a complete drying of the wet core is:
\begin{align}
t_v = \dfrac{\rho_E \, \varepsilon_l}{C_{tot}\, MG_E \, D_{eff} \text{ln} \left(\dfrac{1 - y_{1,int}}{1 - y_{1,0}} \right)}  \, \, \dfrac{r_{core}^2 (3 R_{part} - 2 r_{core})}{6 R_{part}}
\end{align}

\section{Convective velocity}
From the species flux the convective velocity can easily be calculated:
\begin{align}
\dot{n} = v_s \, C_{tot} 
\end{align}
\begin{align}
v_s = \dfrac{D_{eff}}{r^2 \left( \dfrac{1}{r_{core}} - \dfrac{1}{R_{part}}\right)} \, \text{ln} \left( \dfrac{1-y_{1,int}}{1-y_{1,0}} \right)
\end{align}

For Nomenclature see octave documents.
\end{document}
